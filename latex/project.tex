\documentclass{article}
\usepackage{amsmath}
\usepackage[a4paper, margin=3cm]{geometry}
\usepackage{fourier}
\usepackage{inconsolata}
\usepackage{minted}
\usemintedstyle{tango}

\begin{document}

\section{The Fourier Transform}

The Fourier transform is a way of finding constituent frequencies within a function on the real domain.
The Fourier transform extends the concept of the Fourier series (which openerates on a bounded interval) to the real
domain.

\subsection{The Fourier Series}

The Fourier series is defined using the equations described below.
It takes in a single repeating unit of a periodic function and allows us to generate an infinite sum which converges
to the same function. This is useful because the resultant sum ends up depending on trigonometric functions.

Firstly we have a set of numbers, \(A_0\), \(A_n\) and \(B_n\):

\begin{equation}
\begin{aligned}
	A_0 &= \frac{1}{P} \int_{-\frac{P}{2}}^{\frac{P}{2}}
		s(x) \,dx \\
	A_n &= \frac{1}{P} \int_{-\frac{P}{2}}^{\frac{P}{2}}
		s(x) \cos \left(\frac{2\pi nx}{P}\right) \,dx \\
	A_b &= \frac{1}{P} \int_{-\frac{P}{2}}^{\frac{P}{2}}
		s(x) \sin \left(\frac{2\pi nx}{P}\right) \,dx \\
\end{aligned}
\end{equation}

Here \(A_0\) is a constant, and \(A_n\) and \(B_n\) are functions of \(n\).
In fact, since both the denominator of the fraction and the range of the integral for \(A_0\) are the same, \(P\),
\(A_0\) is simply the average around which the function oscillates.

These coefficients allow us to define the Fourier series:

\begin{equation}
	s(x) \sim A_0 + \sum_{n=1}^{\infty} \left(
		A_n \cos \left( \frac{2\pi nx}{P} \right) +
		B_n \sin \left( \frac{2\pi nx}{P} \right)
	\right)
\end{equation}

Here we use a \(\sim\) because this series doesn't always converge to the desired function, although in most cases it
does.

\cite[p. 150]{nixon:2019}
\cite{thompson:1961}
\cite{lestrel:1997}
\cite{cartwright:1990}
\cite{dryden:2016}

\bibliographystyle{ieeetr}
\bibliography{refs}

\section*{Appendix}

\subsection*{main.cpp}

\begin{inputminted}{cpp}{../code/main.cpp}
\end{inputminted}


\end{document}
