\documentclass[12pt,english]{article}

\usepackage[utf8]{inputenc}
\usepackage{amsmath}
\usepackage{cite}
\usepackage{graphicx}
\usepackage[a4paper,bindingoffset=0.2in,left=1.25in,right=.75in,top=1in, bottom = 1in, footskip=.25in]{geometry}
\usepackage{fourier}
\usepackage{inconsolata}
\usepackage{minted}

\renewcommand{\baselinestretch}{1.25}
\graphicspath{ {../images/} }
\usemintedstyle{tango}

\begin{document}
\begin{center}
\includegraphics[scale = 0.5]{LincolnLogo}
\newline\newline
\thispagestyle{empty}
    \textbf{\huge{Classification of Biological Organisms from Images Using Advanced Mathematical Techniques} \\
    [100pt] \Large{Timothy Smith\\25796944} \\
    [40pt]\LARGE{Supervised by Dr Danilo Roccatano}\\
    [50pt]\today \\
	[40pt]Scientific Report in the 3\textsuperscript{rd} Year Module\\ MTH3009 Mathematics Project  }
\end{center}

\newpage
\section*{Abstract}

\newpage
\tableofcontents

\newpage
\section{Introduction}
\section{The Elliptic Fourier Transform}
The Fourier transform is a way of finding constituent frequencies within a function on the real domain.
The Fourier transform extends the concept of the Fourier series (which openerates on a bounded interval) to the real
domain.

\subsection{The Fourier Series}
The Fourier series is defined using the equations described below.
It takes in a single repeating unit of a periodic function and allows us to generate an infinite sum which converges
to the same function. This is useful because the resultant sum ends up depending on trigonometric functions.

Firstly we have a set of numbers, \(A_0\), \(A_n\) and \(B_n\):

\begin{equation}
\begin{aligned}
	A_0 &= \frac{1}{P} \int_{-\frac{P}{2}}^{\frac{P}{2}}
		s(x) \,dx \\
	A_n &= \frac{1}{P} \int_{-\frac{P}{2}}^{\frac{P}{2}}
		s(x) \cos \left(\frac{2\pi nx}{P}\right) \,dx \\
	A_b &= \frac{1}{P} \int_{-\frac{P}{2}}^{\frac{P}{2}}
		s(x) \sin \left(\frac{2\pi nx}{P}\right) \,dx \\
\end{aligned}
\end{equation}

Here \(A_0\) is a constant, and \(A_n\) and \(B_n\) are functions of \(n\).
In fact, since both the denominator of the fraction and the range of the integral for \(A_0\) are the same, \(P\),
\(A_0\) is simply the average around which the function oscillates.

These coefficients allow us to define the Fourier series:
\begin{equation}
	s(x) \sim A_0 + \sum_{n=1}^{\infty} \left(
		A_n \cos \left( \frac{2\pi nx}{P} \right) +
		B_n \sin \left( \frac{2\pi nx}{P} \right)
	\right)
\end{equation}

Here we use a \(\sim\) because this series doesn't always converge to the desired function, although in most cases it
does.

\subsection{The Fourier Transform}
The transform function ($S(t)$) for frequency $f$ is given by
(\ref{fourier:transform}).
\begin{equation} \label{fourier:transform}
	S(t)=\int_{-\infty}^{\infty}s(t)\cdot \mathrm{e}^{-i2\pi ft} dt
\end{equation}

\subsection{Fourier Analysis}

\subsection{Elliptical Fourier Analysis}

\section{Processing the Images}
In order to find shapes in our images for the purposes of analysis,
we will need to detect edges. For this we will use the Canny Edge
detection method, developed by John F. Canny.

\subsection{Convolution Kernels}
The most basic aspect of edge detection is the convolution kernel.
This is an $M{\times}M: \{M=2n+1, n\in\mathbb{N}\}$ matrix
(an odd sided square matrix).
Odd side lengths allow the kernel to be centered at each pixel.

If we let the function $f(x,y)$ represent the original image,
$g(x,y)$ represent the convolved image and $\omega$
represent the kernel, then:
\begin{equation}
	g(x,y)=\sum_{i=-n}^{n}\sum_{j=-n}^{n}\omega(i,j)f(x-i,y-j)
\end{equation}
Notice that the coordinates of the kernel are not $1$ to $M$
as with a traditional matrix, but rather $-n$ to $n$.

The effect of this on an image will be to make each pixel of a
convolved image a function of the surrounding pixels.
The simplest convolution matrix is the identity convolution matrix,
(\ref{convm:identity}).
\begin{equation}
	\label{convm:identity}
	\begin{bmatrix}
		0 & 0 & 0\\
		0 & 1 & 0\\
		0 & 0 & 0\\
	\end{bmatrix}
\end{equation}
More examples include edge detection kernels like
(\ref{convm:edge1}) and (\ref{convm:edge2}).
\begin{center}
\begin{minipage}{.4\linewidth}
\begin{equation} \label{convm:edge1}
	\begin{bmatrix}
		 0 & -1 &  0\\
		-1 &  4 & -1\\
		 0 & -1 &  0\\
	\end{bmatrix}
\end{equation}
\end{minipage}
\begin{minipage}{.4\linewidth}
\begin{equation} \label{convm:edge2}
	\begin{bmatrix}
		-1 & -1 & -1\\
		-1 &  8 & -1\\
		-1 & -1 & -1\\
	\end{bmatrix}
\end{equation}
\end{minipage}
\end{center}

\subsection{Canny Edge Detection}

\section{Conclusion}
Note: these are just example citations.
\cite[p. 150]{nixon:2019}
\cite{thompson:1961}
\cite{lestrel:1997}
\cite{cartwright:1990}
\cite{dryden:2016}
.

\section{Acknowledgements}

\section{Appendix}
Note: These are unlikely to be in the final report, they are simply there as an example appendix for now.
\subsection*{main.cpp}
\inputminted[breakanywhere=true]{cpp}{../code/elliptic_fourier/main.cpp}
\subsection*{elliptic\_fourier.h}
\inputminted[breakanywhere=true]{cpp}{../code/elliptic_fourier/elliptic_fourier.h}
\subsection*{elliptic\_fourier.cpp}
\inputminted[breakanywhere=true]{cpp}{../code/elliptic_fourier/elliptic_fourier.cpp}
\subsection*{drawing\_area.h}
\inputminted[breakanywhere=true]{cpp}{../code/elliptic_fourier/drawing_area.h}
\subsection*{drawing\_area.cpp}
\inputminted[breakanywhere=true]{cpp}{../code/elliptic_fourier/drawing_area.cpp}

\section{Bibliography}
\bibliographystyle{ieeetr}
\bibliography{refs}

\end{document}
